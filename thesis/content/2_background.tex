\chapter{Background}

\section{Wireless Comunication}
\section{\acf{LoRa}}

\section{\acf{LoRaWAN}}

\ac{LoRaWAN} uses the LoRa wireless communication protocol to create a \ac{WAN} where multiple devices can communicate with each other over long distances.

\subsection{\acf{TTN}}

\ac{TTN} is a global community of people building a global open \ac{LoRaWAN} network. \ac{TTN} provides a free \ac{LoRaWAN} network for the public to use. It is based on the \ac{LoRaWAN} protocol and uses the \ac{CUPS} and \ac{LNS} architecture.

% TODO: Explain the CUPS and LNS architecture

\section{Localization}

\subsection{Localization Methods}

\subsubsection{\ac{GPS}}

\subsection{Multiangulation}

\subsubsection{\ac{TDoA}-based}

\subsubsection{\ac{RSSI}-based}

\section{\ac{LoRa} Hardware}

\subsection{Gateways}

A \ac{LoRa} gateway is the device that receives \ac{LoRa} packets from \ac{LoRa} nodes and forwards them to the \ac{LoRaWAN} server.
The gateway is connected to the the internet and can forward packets in two major ways:

\subsubsection{\ac{UDP}}

\subsubsection{\ac{TCP}}

\subsection{Nodes}