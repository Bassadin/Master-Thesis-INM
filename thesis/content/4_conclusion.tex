\chapter{Conclusions}

\section{Comparison of Geolocation Methods}

\subsection{\ac{RSSI}}

\subsection{\ac{ToA} / \ac{TDoA}}

\subsection{Fingerprinting with \ac{kNN}}

\section{Summary}


\section{Outlook}

\subsection{Projects made possible due to new gateways in Furtwangen}

\subsubsection{Measuring the water level of the Breg river}

One of the \ac{LoRaWAN} nodes ordered as part of this thesis is a \emph{Milesight EM310-UDL}, an ultrasonic distance/level sensor.
As the \ac{LoRaWAN} network coverage of Furtwangen is now more than adequate, it would now be possible to use this sensor to measure the water level of the Breg river flowing through the vicinity of the \ac{HFU}.
This would allow for a more accurate prediction of the water level of the Breg river, which might in turn allow for a more accurate prediction of the water level of the Danube river.
Installing this \ac{LoRaWAN} node as well as connecting it to \ac{TTN} and adding an \acf{AS} to it to allow monitoring of the water level of the Breg river would be a good future project for students of the \ac{HFU}, enabled by the gateways placed during this thesis.

\subsubsection{Measuring soil moisture and environmental conditions in the Furtwangen city park}

% TODO: mention Samuels project

\subsection{Further research}

% TODO: if nanosecond level timestamps are available, the ToA method can be used to determine the position of the gateway. This would allow for a more accurate positioning of the gateway, which would in turn allow for a more accurate positioning of the nodes.