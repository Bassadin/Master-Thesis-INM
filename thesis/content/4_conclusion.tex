\chapter{Conclusions}

\section{Comparison of Geolocation Methods}

% TODO compare the geolocation methods

\subsection{Fingerprinting}

\subsection{\acf{RSSI}}

\subsection{\acf{ToA} / \acf{TDoA}}

% TODO: not feasible due to lack of nanosecond level timestamps as well as the fact that the gateways' timestamps are not synchronized with the nodes' timestamps in all cases

\section{Summary}

% Mention problems with falsely entered gateway locations and how they could be fixed

\subsection{Comparison of findings with existing work}

% TODO mention the existing work in conjunction with my own findings as far as localization is concerned

\section{Outlook}

% TODO: What else can be done? What are the next steps?

\subsection{Projects made possible due to new \acs{LoRaWAN} gateways in Furtwangen}

As there are now several new \ac{LoRaWAN} gateways in Furtwangen, there are several projects that can be done in the future that would not have been possible before without installing such gateways by oneself.

\subsubsection{Measuring the water level of the Breg river}

One of the \ac{LoRaWAN} nodes ordered as part of this thesis is a \emph{Milesight EM310-UDL}, an ultrasonic distance/level sensor.
As the \ac{LoRaWAN} network coverage of Furtwangen is now more than adequate, it would now be possible to use this sensor to measure the water level of the Breg river flowing through the vicinity of the \ac{HFU}.
This would allow for a more accurate prediction of the water level of the Breg river, which might in turn allow for a more accurate prediction of the water level of the Danube river.
Installing this \ac{LoRaWAN} node as well as connecting it to \ac{TTN} and adding an \acf{AS} to it to allow monitoring of the water level of the Breg river would be a good future project for students of the \ac{HFU}, enabled by the gateways placed during this thesis.

\subsubsection{Measuring soil moisture and environmental conditions in the Furtwangen city park}

% TODO: mention Samuels project
% TODO German? English?
Samuel Kasper, ein Student der Fakultät \ac{DM} der \ac{HFU} schrieb in diesem Semester seine Bachelor-thesis
Das Aufstellen der \ac{LoRaWAN} Gateways half ihm dabei, indem er dadurch nicht seine eigenen Gateways aufstellen musste und stattdessen die nun bestehende Infrastruktur im Ort verwenden konnte.

\subsection{Further research}

% TODO: if nanosecond level timestamps are available, the ToA method can be used to determine the position of the gateway. This would allow for a more accurate positioning of the gateway, which would in turn allow for a more accurate positioning of the nodes.