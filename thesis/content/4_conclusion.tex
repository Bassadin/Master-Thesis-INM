\chapter{Conclusions}

\section{Gateway Locations}

% Evaluate the chosen gateway locations and give example of the ranges, etc.

% Also add graphs of recorded gateway ranges from TTN Mapper?

\section{Collected data in Furtwangen on TTN Mapper}

% TODO: add before and after images of TTN Mapper heatmap

\subsection{\acl{TTNM} heatmap view after additional data collection}\label{sec:ttm_heatmap_after}

% TODO add after image of TTN Mapper heatmap

\section{Comparison of Geolocation Methods}

% TODO compare the geolocation methods

\subsection{\acf{RSSI}}

% TODO: mention the problems with RSSI-based geolocation (different antennas per gateway, different antennas per node, different environments, etc.)

\subsection{\acf{ToA} / \acf{TDoA}}

% TODO: not feasible due to lack of nanosecond level timestamps as well as the fact that the gateways' timestamps are not synchronized with the nodes' timestamps in all cases

\subsection{Fingerprinting}

% TODO: describe how this worked and how it could be improved (worked pretty okay but has only limited accuracy)

% TODO: mention the potential of using machine learning to improve the fingerprinting method

\section{Comparison of findings with existing work}

% TODO mention the existing work in conjunction with my own findings as far as localization is concerned

\section{Problems}

% Mention problems with falsely entered gateway locations and how they could be fixed

% TODO write some more about this
As mentioned in \cref{sec:spreading-factors}, using a different \ac{SF} on the end devices can also impact the accuracy of the geolocation methods because they influence the \ac{RSSI} values.

\section{Outlook}

% TODO: What else can be done? What are the next steps?

\subsection{Shortcomings of the \acf{TTNL} software}

% TODO: show what else could be done - automatically recognize if gateways moved too far away like in TTN Mapper etc.

\subsection{Projects made possible due to new \acs{LoRaWAN} gateways in Furtwangen}

As there are now several new \ac{LoRaWAN} gateways in Furtwangen, there are several projects that can be done in the future that would not have been possible before without installing such gateways by oneself.

This section will list some of these projects and describe how they could be implemented.

\subsubsection{Measuring the water level of the Breg river}

One of the \ac{LoRaWAN} nodes ordered as part of this thesis is a \emph{Milesight EM310-UDL}, an ultrasonic distance/level sensor.
As the \ac{LoRaWAN} network coverage of Furtwangen is now adequate, it would be possible to use this sensor to measure the water level of the Breg river flowing through the vicinity of the \ac{HFU}.
This would allow for a more accurate prediction of the water level of the Breg river, which might in turn allow for a more accurate prediction of the water level of the Danube river.
Installing this \ac{LoRaWAN} node as well as connecting it to \ac{TTN} and adding an \acf{AS} to it to allow monitoring of the water level of the Breg river would be a good future project for students of the \ac{HFU}, enabled by the gateways placed during this thesis.

\subsubsection{Measuring soil moisture and environmental conditions in the Furtwangen city park}

This same semester, Samuel Kasper, a student of the \ac{DM} faculty of the \ac{HFU} wrote his bachelor thesis about measuring the humidity as well as the temperature of the soil to monitor plants and crop growth.
The installation of \ac{LoRaWAN} gateways in the city of Furtwangen helped him in this regard, as he did not have to install his own gateways and instead could use the now existing infrastructure in the city.

\subsection{Further research}

\subsubsection{Improving timestamp accuracy for \acf{ToA} geolocation method}

% TODO: if nanosecond level timestamps are available, the ToA method can be used to determine the position of the gateway. This would allow for a more accurate positioning of the gateway, which would in turn allow for a more accurate positioning of the nodes.

\subsubsection{Using \acf{ML} to improve the fingerprinting method}