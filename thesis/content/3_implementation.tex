\chapter{Implementation}

\section{Installing gateways in Furtwangen}

Since, at the beginning of this thesis, there were only two gateways installed in the vicinity of Furtwangen, the first step was to research, order and then install additional gateways in the city.

The gateways that were used in this thesis were mainly ones that were already in the possession of the \ac{HFU}.
Such was the case for the two \emph{Dragino LG308} indoor gateways as well as a \emph{RAK7258} indoor gateway.

Additionally, for development purposes, a \emph{Dragino PG1301} outdoor gateway was also used.
The PG1301 is designed to be used with a Raspberry Pi and mounts on top of it in typical Raspberry Pi shield fashion.

\subsection{Researching gateways}

One notable exception are the two MikroTik wAP LR8 kit gateways that was purchased for the thesis.
It was chosen after recommendation by a member of the \ac{HSN-TTN} team~\cite{noauthor_eingesetzte_nodate}.
The MikroTik wAP LR8 kit was a compelling offer due to its low price and the fact that it comes with MikroTik's \emph{RouterOS}, a well-established GUI for configuring networking devices.
A picture of this gateway can be seen in \cref{pic:mikrotik-lr8-kit-gateway}.
The antenna used with the MikroTik wAP LR8 kit is also from MikroTik, which is a 6.5 dBi antenna which can be seen in \cref{pic:mikrotik-antenna-c-building}.

The other gateway that was purchased for the thesis was a \emph{Dragino DLOS8N} outdoor gateway.
This gateway was chosen because of difficulties in ordering the second MikroTik wAP LR8 kit.
It was available right away at the time of ordering and was thus chosen as a short-term replacement.
Still, it was used anyway as the user interface is identical to the one used in the Dragino LG308 gateway which made it easy to set up.
The DLOS8N also has a case that is weatherproof and made for outdoor use.

\subsection{Locations}

% TODO: Add map with Gateway locations

Multiple locations proved themselves to be worth considering when it came to installing new \ac{LoRaWAN} gateways in Furtwangen.
In the following sections, the chosen locations are described and evaluated.

\subsubsection{\ac{HFU} C building}

The first antenna and gateway installed was a Dragino LG308 connected to a MikroTik antenna on top of the \ac{HFU} C building as seen in \cref{pic:mikrotik-antenna-c-building}.
While not as high in altitude as the \ac{GHB} building, it was thought to still be a good location to receive signals from the surrounding area.
The roof of the C building is located in an exposed in the middle of the valley Furtwangen is situated in.

\subsubsection{\acf{GHB} and \acf{ASH} student dormitories}

The \ac{GHB} is a popular student dormitory located on the Großhausberg mountainside in Furtwangen.
Its exposed location and the fact that it is among the highest buildings in the city makes it a good location for putting up a \ac{LoRaWAN} gateway.

The network infrastructure of both the \ac{GHB} and the \ac{ASH} are managed by the ``\ac{GHB} netadmins'' who are students of \ac{HFU}.
Thus, with their help, it was possible to install antennas on the roofs of the \ac{GHB} and \ac{ASH} student dormitories.

\subsubsection{Brend tower}

% todo: explain talks with EGT GmbH and problems associated with deployment (private network, etc.)

\section{Collecting additional \ac{TTNM} data in the Furtwangen area}

In order to have more data to work with as far as geolocation calculations are concerned, there was a need to collect more \ac{LoRaWAN} \ac{RSSI} and \ac{GPS} data that could be synthesized.

\subsection{Used \ac{LoRa} nodes}

In this chapter, the \ac{LoRa} nodes used for collecting \ac{GPS} data for \ac{TTNM} are described are briefly evaluated.

\subsection{ELV LoRIS}

% worked the best out of the box, great manuals, etc.

The ELV LoRIS was the \ac{GPS} tracker that worked best out of the box as far as general usability is concerned.
It has a USB-C port, which made powering the tracker with an external power bank simple.


\subsection{ELV \ac{GPS} Tracker}

% Problem: needed to get a 9V battery attached to it (no battery holder included)
% 5-40V input voltage is very flexible

\subsection{Heltec HTCC-AB02S}

% didn't work well enough without additional GPS antenna
% screen is pretty nice, though

% TODO
\subsection{dnt-LW-ATS}

% TODO mention problems with the node (configuration, etc.)

\section{Program structure}

\subsection{Overview}

% TODO add diagram of program structure

\subsection{Used technologies for the backend}

% TODO

\subsubsection{Node.js}

\subsubsection{Typescript}

\subsubsection{Express}

\subsubsection{Prisma}

\subsubsection{PostgreSQL}

\subsubsection{Jest for testing}

\subsection{Used technologies for the frontend}

\subsubsection{Vue.js}