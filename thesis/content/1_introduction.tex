\chapter{Introduction}

\section{Motivation}

The global market for IoT devices is expected to grow from 397.62 B USD in 2019 to 1,110 B USD by 2028~\cite{grand_view_research_global_2022}.
Over short ranges, it is relatively easy to connect devices to the internet wirelessly with technologies such as ZigBee.
Bridging long distances, however, can become a challenge if devices are also to consume as little power as possible.
The wireless communication protocol \ac{LoRa} promises to address this challenge by offering a vast range with the drawback of having very little throughput.

While \ac{LoRa} end devices can feature \ac{GPS} receivers to geolocate them when deployed, they use additional space and, more importantly, power.
This is especially problematic for battery-powered devices that are deployed in remote areas.
Using the available \ac{LoRa} Gateways, this thesis aims to compare methods to geolocate \ac{LoRa} devices without the need for a GPS receiver.

\section{Research Questions}

\begin{enumerate}
    \item In what ways can \ac{LoRa} devices be geolocated without the need for a GPS receiver?
    \item How do these ways compare to each other?
    \item What factors can influence the accuracy of these kinds of geolocation?
\end{enumerate}

\section{Existing Work}

\section{Outline}