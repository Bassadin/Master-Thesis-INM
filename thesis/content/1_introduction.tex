\chapter{Introduction}

\section{Motivation}

The worldwide market for \ac{IoT} devices is expected to increase from USD 686.46 billion in 2023 to USD 3,352.97 billion in 2030~\cite{statista_industrial_2023}.
An increasing number of industries are utilizing \ac{IoT} devices to streamline workflows.
Established technologies, such as Zigbee, Thread, or Wi-Fi, can be used to wirelessly connect to the internet over short ranges for mains-powered devices, or devices that do not require long-lasting battery life.
While \ac{LTE} can be used to bridge long distances, it is unsuitable for devices that need to conserve power.

The \ac{LoRa} wireless communication protocol addresses this challenge by providing a long-range and low-power usage solution, despite having limited throughput capabilities.
Under ideal conditions, the longest transmission range of a \acl{LRED} has been measured at 832 km~\cite{the_things_network_global_team_lora_nodate}.
While \aclp{LRED} usually reach much shorter distances in reality, they can still effectively transmit data over multiple kilometers while consuming little power.
In the first half of 2020, \ac{LoRa} was the world's second most popular \ac{LPWAN} technology, with a usage rate of 41\%.
It ranked just behind NB-IoT with 44\%~\cite{iot_analytics_lpwa_2020}.

Although \aclp{LRED} can include \ac{GNSS} receivers for geolocation purposes, they consume a significant amount of power, particularly during location acquisition.
Battery-powered devices deployed in remote areas are expected to operate for several years on a single battery; therefore, this is especially problematic in this scenario.
Geolocation of \aclp{LRWED} is also possible without the use of a \ac{GNSS} receiver by utilizing the signals they send to the \aclp{LRWGW} in the network.

This thesis aims to evaluate and compare \ac{GNSS}-less methods for the geolocation of unaltered \aclp{LRWED} and \aclp{LRWGW}.
The aim is to evaluate methods that can be expanded to an entire \ac{LoRaWAN} network.
This thesis does not focus on indoor or outdoor localization exclusively, but it aims to provide a general overview of the topic.

To achieve this objective, \aclp{LRWGW} already registered in \acf{TTN} in and around Furtwangen were used.
Additionally, new \aclp{LRWGW} were installed in that area.

\section{Research Questions}\label{sec:introduction-research-questions}

\begin{enumerate}
      \item Can \aclp{LRWED} be geolocated without a \ac{GNSS} receiver?
      \item How do methods for such \ac{GNSS}-less geolocation compare?
      \item Which factors can impact the accuracy of this type of geolocation?
\end{enumerate}

\section{Related Work}\label{sec:related-work}
% NOTE: Zahoransky: Should be about 1-3 (PDF) pages

Some papers have already looked into the topic of geolocation of \aclp{LRED} and the geolocation of \ac{IoT} devices in general.

In 2004, Elnahrawy et al.\ researched the limits of localization using signal strength~\cite{elnahrawy_limits_2004}.
While they used Wi-Fi instead of \ac{LoRa}, their findings are still relevant.
During their research, they concluded that indoor, building-scale \ac{RSS}-based localization without any additional complex environmental models has a median localization error of \SI{3.04}{\meter} and a 97th percentile error of \SI{9.14}{\meter}.

Whitehouse et al.\ researched the localization of sensor nodes in 2007~\cite{whitehouse_practical_2007}.
They used \ac{RSS} values to determine the distances between nodes, and achieved an error of \SI{4.1}{\meter} in their network consisting of 49 nodes.
The scale of that network only covered around \SI{30}{\meter} by \SI{30}{\meter}, making it difficult to apply their results to larger real-world scenarios.

Fargas and Petersen describe and evaluate a \ac{ToA}-based multilateration geolocation method in their paper \acs{GPS}-free geolocation using \acs{LoRa} in low-power \acsp{WAN} released in 2017~\cite{fargas_gps-free_2017}.
However, they only deployed a small \ac{LoRaWAN} network with 4 \aclp{GW} and do not evaluate the method in a larger network such as \ac{TTN}.

In 2018, Magno et al.\ showed a promising approach to locate \aclp{LRED}~\cite{magno_poster_2018}.
They combined the \acf{RTK} technique with \ac{LoRa} to achieve an accuracy of \SI{20}{\centi\meter} in some experimental results.
\ac{RTK} is a technique that uses a fixed, time-synchronized base station with \ac{GPS} to improve the localization signal of a mobile device called Rover.

Gu et al.\ researched \ac{LoRa}-based localization in 2018.
They concluded that both \ac{ToA} and \ac{RSSI} based methods are flawed and are not accurate enough for any tasks that require precision, such as steering of autonomous vehicles and similar use cases~\cite{gu_lora-based_2018}.

Aernouts et al.\ used a fingerprinting approach in 2018~\cite{aernouts_sigfox_2018}.
They compared the accuracy of fingerprinting localization for \ac{LoRaWAN} and SIGFOX and concluded that \ac{LoRaWAN} is more accurate than SIGFOX.
SIGFOX is another \ac{LPWAN} technology similar to \ac{LoRaWAN}.
It uses a different modulation technique to transmit the physical signal.
In urban environments, the accuracy was \SI{688.97}{\meter} for their SIGFOX dataset and \SI{398.40}{\meter} for their LoRaWAN dataset.

In another example, Lam et al.~proposed a method to geolocate \aclp{LRED} using an algorithm that is also based on \ac{RSS} values~\cite{lam_new_2018}.
They focused specifically on noise affecting \aclp{LRED} and \aclp{LRGW} in their research.
As a result, they proposed an algorithm called Localization Linear Model with Density-based Clustering which can select non-noisy \aclp{LRED} to geolocate \aclp{LRED}.
Using this algorithm, they achieve an accuracy of \SI{7.51}{\meter}.

Mackey et al.~used \aclp{LRGW} over relatively short distances to geolocate \aclp{LRED} by using their \ac{RSS} values~\cite{mackey_lora-based_2019}.
They used a soccer field as their test environment and achieved an accuracy of \SIrange{9}{20}{\meter}.
Notably, \ac{LoS} was a given in their test environment.

Anagnostopoulos and Kalousis compared some fingerprinting based localization methods for \ac{LoRaWAN} in 2019~\cite{anagnostopoulos_reproducible_2019}.
They used methods like \ac{kNN} as well as a neural network approach.
Using the \ac{kNN} approach, they achieved an accuracy of \SIrange{273}{394}{\meter}.

Anjum et al.\ also evaluated \ac{RSSI} fingerprinting in 2019~\cite{anjum_analysis_2019}.
In their research, they compared \ac{LoS} and \ac{NLoS} scenarios, analyzing the path loss exponent and the standard deviation of the shadowing \ac{MPP} parameter in each of these environments.
They did not use the \ac{LoRaWAN} protocol but instead used a small test setup of their own \aclp{LRED}.
Placing nodes in different stories of a building, they measured a varying accuracy of \ac{RSSI} to distance mappings.

Fernandes et al.\ used a hybrid approach to locate elderly care recipients indoors: \ac{UWB} technology for localization and \ac{LoRaWAN} to send data to their caregivers~\cite{fernandes_hybrid_2020}.
They chose the \ac{LoRaWAN} technology to transmit the actual localization data for its ability to be implemented as both public and private network infrastructures.

Jeftenić et al.\ researched the impact of environmental parameters such as precipitation, temperature, and humidity on \ac{SNR} and \ac{RSS} in \ac{LoRaWAN} in 2020~\cite{jeftenic_impact_2020}.
Their results included a high negative impact of rising temperature and relative humidity on the \ac{RSS} and \ac{SNR} values.
Snowing also led to fluctuations in \ac{RSS} and \ac{SNR} values.

In January 2023, Perković et al.\ published a paper on the topic of \ac{LoRa}-based indoor localization.
They used a machine learning approach in addition to \ac{RSSI} and \ac{SNR} values to improve the accuracy of their localization method~\cite{perkovic_machine_2023}.

In conclusion, several papers have already examined the topic of \ac{LoRa}-based localization and related areas.
This thesis will utilize the conclusions of these papers as the basis for applying them to a real-world deployment scenario in Furtwangen.

\section{Outline}

This thesis is structured as follows:

\begin{enumerate}
      \item \textbf{Background}:
            This chapter provides background information on topics that are necessary to understand the main part of the thesis, such as some basics of \ac{RF} technology, \ac{LoRa}, \ac{LoRaWAN} and \ac{GNSS} as well as geolocation techniques such as multilateration with different value to range mappings and fingerprinting.
      \item \textbf{Implementation}:
            The topic of this chapter is how the geolocation of \acl{LRWED} was implemented in practice.
            It also describes what hardware was used and where new \aclp{LRWGW} were deployed.
            The architecture of the \ac{TTNL} web application, which was built to perform calculations on the collected data as well as to visualize it, is also shown.
      \item \textbf{Conclusion}:
            This section serves as both a conclusion to the thesis and an evaluation of the topics discussed.
            The placement of \aclp{LRWGW} in Furtwangen is evaluated and the accuracy of the geolocation methods is discussed.
            Furthermore, an evaluation of the technology choices made in relation to the \ac{TTNL} application is presented.
            The section concludes with a discussion of challenges encountered and potential solutions for overcoming them in the future.
\end{enumerate}

\section{Expression of Thanks}\label{sec:expression-of-thanks}

I would like to acknowledge the assistance I received during the creation of this thesis from various parties.

My supervising professor \textbf{Prof.\ Dr.\ Richard Zahoransky}, for his support and guidance throughout the thesis.

The \textbf{Netzint GmbH} in Gütenbach and especially \textbf{Mr.\ Tom Lehmann} for describing the implementation of \ac{LoRaWAN} technology in the infrastructure of a company in Furtwangen.
He also helped me to get in touch with Mr.\ Odin Jäger, who allowed me to install a \acl{LRWGW} on the roof of the \ac{HFU} O building.

The \textbf{EGT GmbH} from Triberg, for giving me some insight into their own efforts with \ac{LoRaWAN} technology.

The \textbf{\acl{GHB-NA}} team, who spent many hours helping me install \aclp{LRWGW} within the buildings they manage as well as configuring the external network access needed to control and configure it.
Its members include, but are not limited to: Kai-Luka Lüthje, Simon Blaßdörfer, Valentin Weber, Florian Brunen, and Fabian Hurst.

\textbf{Mr.\ Ralf Schellhammer}, for providing me with the necessary hardware to perform the experiments.
His motivation in explaining technical intricacies of radio transmissions and his willingness to help me with installing some \aclp{LRWGW} was invaluable.
He also helped with the installation of new \aclp{LRWGW} in the vicinity of Furtwangen on several occasions.

The Amateur Radio Club of Furtwangen, \textbf{\emph{DL0FIS}}, for allowing me to install a \acl{LRWGW} on their mast near Furtwangen.
Mr.\ Ralf Schellhammer also helped me to get in contact with them.

\textbf{Mr.\ Martin Kramer}, for configuring the \ac{HFU} network infrastructure I used for some \aclp{LRWGW}.

\textbf{Mr.\ Michael Lavalle}, member of the \ac{HSN-TTN} project, who helped me to collect more data in the Furtwangen area.
He also provided some valuable feedback on which \aclp{LRWGW} to buy.

