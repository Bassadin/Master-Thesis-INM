\chapter{Introduction}

\section{Motivation}

The global market for \ac{IoT} devices is expected to grow from USD 686.46 billion in 2023 to USD 3,352.97 billion in 2030~\cite{statista_industrial_2023}.
More and more industries are adopting \ac{IoT} devices in order to optimize their workflows.
Over short distances and with mains-powered devices or devices that don't require years of battery life, it is relatively easy to achieve connections to the internet wirelessly with established technologies such as ZigBee, Thread or Wi-Fi.
However, bridging long distances can become a challenge when devices also need to consume as little power as possible.

The \ac{LoRa} wireless communication protocol promises to address this challenge by offering a long range and low power usage with the drawback of having little throughput.
The longest transmission range of a \ac{LoRa} device, under ideal conditions, has been measured at 832 km~\cite{the_things_network_global_team_lora_nodate}.
While, realistically, \ac{LoRa} devices usually reach much shorter distances, they can still be used to connect devices over long distances of multiple kilometers while consuming little power.
In the first half of 2020, \ac{LoRa} was the second most used \ac{LPWAN} technology worldwide, scoring 41\% and ranking just behind NB-IoT with 44\%~\cite{iot_analytics_lpwa_2020}.

While \ac{LoRa} end devices can feature \ac{GPS} receivers to geolocate them when deployed, they consume considerable amounts of power, especially when trying to obtain a location fix.
This is particularly problematic for battery-powered devices that are deployed in remote areas and are expected to last several years on a single battery.
Unlike \ac{GPS}, \ac{LoRa} nodes may also be geolocated without a \ac{GPS} receiver by using the signals they send to the \ac{LoRa} gateways in the network.

The goal of this thesis is to analyze and compare \ac{GPS}-less methods for geolocating standard \ac{LoRaWAN} nodes without modifying them or the gateways used.
The goal is not to focus specifically on indoor or outdoor localization specifically but to provide a general overview of the topic.

To achieve this goal, some \ac{LoRaWAN} gateways that were already registered in \acf{TTN} in Furtwangen were used as well as new gateways placed in the Furtwangen area.

\section{Research Questions}\label{sec:introduction-research-questions}

\begin{enumerate}
      \item Is it possible for \ac{LoRa} devices to be geolocated without the need for a \ac{GPS} receiver?
      \item How do methods for such \ac{GPS}-less geolocation compare?
      \item What factors can affect the accuracy of this type of geolocation?
\end{enumerate}

\section{Existing / Related Work}\label{sec:related-work}
% NOTE: Zahoransky: Should be about 1-3 (PDF) pages

Some papers have already looked into the topic of geolocation of \ac{LoRa} devices and the geolocation of \ac{IoT} devices in general.

In 2004, Elnahrawy et al.\ researched ``The Limits of Localization Using Signal Strength''~\cite{elnahrawy_limits_2004}.
While they used Wi-Fi instead of \ac{LoRa}, their findings are still relevant.
During their research, they concluded that indoor, building-scale \ac{RSS}-based localization without any additional complex environmental models or additional localization infrastructure has a median localization error of \SI{3.04}{\meter} and a 97th percentile error of \SI{9.14}{\meter}.

Magno et al.\ showed a promising approach to locate \ac{LoRa} devices in 2018~\cite{magno_poster_2018}.
They combined the \acf{RTK} technique with \ac{LoRa} to achieve an accuracy of \SI{20}{\centi\meter} in some experimental results.
\ac{RTK} is a technique that uses a fixed, time-synchronized base station with \ac{GPS} to improve the localization signal of a mobile device called Rover.

Gu et al.\ researched \ac{LoRa}-based localization in 2018.
They concluded that both \ac{ToA} and \ac{RSSI} based methods are flawed and are not accurate enough for any task that requires precision, such as steering of autonomous vehicles and similar use cases~\cite{gu_lora-based_2018}.

Aernouts et al.\ used a fingerprinting approach in 2018~\cite{aernouts_sigfox_2018}.
They compared the accuracy of fingerprinting localization for \ac{LoRaWAN} and \emph{SIGFOX} and concluded that \ac{LoRaWAN} is more accurate than \emph{SIGFOX}.
\emph{SIGFOX} is another \ac{LPWAN} technology similar to \ac{LoRaWAN}.
It uses a different modulation technique to transmit the physical signal.
In urban environments, the accuracy was \SI{688.97}{\meter} for their \emph{SIGFOX} dataset and \SI{398.40}{\meter} for their LoRaWAN dataset.

In another example, Lam et al.~proposed a method to geolocate \ac{LoRa} devices using a new kind of algorithm that is also based on \ac{RSS} values~\cite{lam_new_2018}.
They focused specifically on noise affecting anchor nodes/gateways in their research.
As a result, they proposed an algorithm called \emph{Localization Linear Model with Density-based Clustering} which can select non-noisy nodes to geolocate \ac{LoRa} devices.
Using this algorithm, they achieve an accuracy of \SI{7.51}{\meter}.

Mackey et al.~used \ac{LoRa} gateways over relatively short distances to geolocate \ac{LoRa} devices by using their \ac{RSS} values~\cite{mackey_lora-based_2019}.
They used a soccer field as their test environment and achieved an accuracy of \SIrange{9}{20}{\meter}.
Notably, \ac{LoS} was a given in their test environment.

Anagnostopoulos and Kalousis compared some fingerprinting based localization methods for \ac{LoRaWAN} in 2019~\cite{anagnostopoulos_reproducible_2019}.
They used methods like \ac{kNN} as well as a neural network approach.
Using the \ac{kNN} approach, they achieved an accuracy of \SIrange{273}{394}{\meter}.

Anjum et al.\ also evaluated \ac{RSSI} fingerprinting in 2019~\cite{anjum_analysis_2019}.
In their research, they compared \ac{LoS} and \ac{NLoS} scenarios, analyzing the path loss exponent and the standard deviation of the shadowing \ac{MPP} parameter in each of these environments.
They did not use the \ac{LoRaWAN} protocol but instead used a small test setup of their own \ac{LoRa} nodes.
Placing nodes in different stories of a building, they measured a varying accuracy of \ac{RSSI} to distance mappings.

Fernandes et al.\ used a hybrid approach to locate elderly care recipients indoors: \ac{UWB} technology for localization and \ac{LoRaWAN} to send data to their caregivers~\cite{fernandes_hybrid_2020}.
They chose the \ac{LoRaWAN} technology to transmit the actual localization data for its ability to be implemented as both public and private network infrastructures.

Jeftenić et al.\ researched the impact of environmental parameters such as precipitation, temperature, and humidity on \ac{SNR} and \ac{RSS} in \ac{LoRaWAN} in 2020~\cite{jeftenic_impact_2020}.
Their results included a high negative impact of rising temperature and relative humidity on the \ac{RSS} and \ac{SNR} values.
Snowing also led to fluctuations in \ac{RSS} and \ac{SNR} values.

In January 2023, Perković et al.\ published a paper on the topic of \ac{LoRa}-based indoor localization.
They used a machine learning approach in addition to \ac{RSSI} and \ac{SNR} values to improve the accuracy of their localization method~\cite{perkovic_machine_2023}.

In conclusion, some papers have already researched the topic of \ac{LoRa}-based localization and adjacent topics.
This thesis will build upon the findings of these papers and try to apply them to a real-world deployment scenario in the city area of Furtwangen.

\section{Outline}

This thesis is structured as follows:

\begin{enumerate}
      \item \textbf{Background}:
            This chapter provides background information on topics that are necessary to understand the main part of the thesis, such as some basics of \ac{RF} technology, \ac{LoRa}, \ac{LoRaWAN} and \ac{GPS} as well as geolocation techniques such as multilateration and fingerprinting.
      \item \textbf{Implementation}:
            The topic of this chapter is how the geolocation was implemented in practice.
            It also describes what hardware was used and where new gateways were deployed.
            The development of the \ac{TTNL} web application, which was built to perform calculations on the collected data as well as to visualize it, is also shown.
      \item \textbf{Conclusion}:
            This chapter concludes the thesis and gives an evaluation on the topics discussed in it.
            The placement of gateways in the Furtwangen city area is evaluated and the accuracy of the geolocation methods is discussed.
            Technology choices related to the \ac{TTNL} application are also evaluated.
            Some problems and challenges encountered and how they could be overcome in the future are also discussed.
\end{enumerate}

% QUESTION does this belong here?

\section{Expression of Thanks}\label{sec:expression-of-thanks}

I would like to express my gratitude to several parties for the help they provided me with during the creation of this thesis:

My supervising professor Prof.\ Dr.\ Richard Zahoransky, for his support and guidance throughout the thesis.
He helped me to channel my motivation for the topic into productivity and provided me with valuable feedback.

The Netzint GmbH in Gütenbach and especially Mr.\ Tom Lehmann for describing the implementation of the \ac{LoRaWAN} technology in the infrastructure of a company in Furtwangen.
He also helped me to get in touch with Mr.\ Odin Jäger, who allowed me to install a gateway on the roof of the \ac{HFU} O building.

The EGT GmbH from Triberg, for giving me some insight into their own efforts with the \ac{LoRaWAN} technology.

The \ac{GHB} NetAdmins team, who spent many hours helping me install hardware within the building infrastructure they manage as well as configuring the external network access needed to control and configure it.
Its members include, but are not limited to: Kai-Luka Lüthje, Simon Blaßdörfer, Valentin Weber, Florian Brunen, and Fabian Hurst.

Mr.\ Ralf Schellhammer, for providing me with the necessary hardware to perform the experiments.
His motivation in explaining technical intricacies of radio transmission and his willingness to help me with installing some \ac{LoRaWAN} gateways was invaluable.
He also helped with the installation of new gateways in the vicinity of Furtwangen on several occasions.

Mr.\ Martin Kramer, for configuring the network I used for the gateways as well as connecting them to the \ac{HFU} network infrastructure.

Mr.\ Michael Lavalle of the \ac{HSN-TTN} project, who helped me to collect more data in the Furtwangen area.
He also provided some valuable feedback on which gateways to buy.

The Amateur Radio Club of Furtwangen, \emph{DL0FIS}, for allowing me to install a gateway on their mast near Furtwangen.
