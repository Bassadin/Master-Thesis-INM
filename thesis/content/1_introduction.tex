\chapter{Introduction}

\section{Motivation}

The global market for \ac{IoT} devices is expected to grow from 397.62 B USD in 2019 to 1,110 B USD by 2028~\cite{grand_view_research_global_2022}.
Over short ranges and with mains-operated devices, it is relatively easy to achieve connections to the internet wirelessly with technologies such as ZigBee, Thread of Wi-Fi.
Bridging long distances, however, can become a challenge if devices are also required to be autonomous which means to consume as little power as possible.

The wireless communication protocol \ac{LoRa} promises to address this challenge by offering a vast range with the drawback of having very little throughput.
The longest transmission range of a \ac{LoRa} device, under ideal conditions, has been measured at 832 km~\cite{the_things_network_global_team_lora_nodate}.
While, realistically, such devices usually reach much shorter distances, they can still be used to connect devices over long distances of multiple kilometers.

While \ac{LoRa} end devices can feature \ac{GPS} receivers to geolocate them when deployed, they use additional space and, more importantly, power.
This is especially problematic for battery-powered devices that are deployed in remote areas.
Contrary to \ac{GPS}, \ac{LoRa} nodes may also be geolocated without a \ac{GPS} receiver by using the signals they send to the \ac{LoRa} Gateways in the network.

This thesis aims to analyze and compare \ac{GPS}-less methods to geolocate off-the-shelf \ac{LoRaWAN} nodes without modifying them or the gateways used.
The goal is not to focus on indoor or outdoor applications specifically but to provide a general overview of the topic.
To achieve this, available \ac{LoRa} Gateways that are registered in \acf{TTN} as well as new ones placed in the vicinity of Furtwangen were used.

\section{Research Questions}

\begin{enumerate}
      \item In what ways can \ac{LoRa} devices be geolocated without the need for a \ac{GPS} receiver?
      \item How do these ways compare to each other?
      \item What factors can influence the accuracy of these kinds of geolocation?
\end{enumerate}

\section{Existing / Related Work}

Some papers have already looked into the topic of geolocation of \ac{LoRa} devices.
Mackey et al.~used \ac{LoRa} gateways over relatively short distances to geolocate \ac{LoRa} devices by using their \ac{RSSI} values~\cite{mackey_lora-based_2019}.

In another example, Lam et al.~proposed a method to geolocate \ac{LoRa} devices using a new kind of algorithm that is also based on \ac{RSSI} values~\cite{lam_new_2018}.

Gu et al.\ researched \ac{LoRa}-based localization in 2018.
They concluded that both \ac{ToA} and \ac{RSSI} based methods are flawed and are not accurate enough for any task that requires precision, such as steering of autonomous vehicles and similar use cases~\cite{gu_lora-based_2018}.

This year in January, Perković et al.\ published a paper on the topic of \ac{LoRa}-based indoor localization.
In addition to \ac{RSSI} and \ac{SNR} values, they used machine learning to improve the accuracy of their localization~\cite{perkovic_machine_2023}.

% TODO: Mention 3 or 4 more, there's plenty of them

\section{Outline}

This thesis is structured as follows:

\begin{enumerate}
      \item \textbf{Introduction}:
            This chapter briefly states the topic of this thesis and the research questions.
      \item \textbf{Background}:
            This chapter provides background information on topics that are necessary to understand the main part of the thesis, such as \ac{RF} technology, \ac{LoRa}, \ac{LoRaWAN} and \ac{GPS} as well as geolocation techniques.
      \item \textbf{Implementation}:
            This chapter presents the results of the geolocation experiments and how they were implemented in practice.
            It also describes what hardware was used, where new gateways were deployed in what ways and how the \ac{TTNL} application was developed to perform calculations on the collected data.
      \item \textbf{Conclusion}:
            This chapter concludes the thesis and gives an evaluation on the topics discussed in it.
            It also talks about some problems and challenges that occurred and how they could be overcome in the future.
\end{enumerate}

% TODO does this belong here?

\section{Expression of Thanks}

I would like to express my thanks to several parties for the help they provided me with during the creation of this thesis:

My supervising professor Prof.\ Dr.\ Richard Zahoransky, for his support and guidance throughout the thesis.
He helped me channel my motivation for the topic into productiveness and provided me with valuable feedback.

The Netzint GmbH in Gütenbach and Mr.\ Tom Lehmann in specific, for describing the implementation of \ac{LoRaWAN} technology into a company's infrastructure in Furtwangen to me.
He also helped me to establish contact with Mr.\ Odin Jäger who allowed me to put up a gateway on the \ac{HFU} O building.

The EGT GmbH from Triberg, for providing some insight into their own endeavors with \ac{LoRaWAN} technology.

The \ac{GHB} NetAdmins team, who spent many hours helping me install hardware within the building infrastructure they manage as well as configuring the external network access needed to control and configure it.
Their members include, but are not limited to: Kai-Luka Lüthje, Simon Blaßdörfer, Valentin Weber, Florian Brunen, and Fabian Hurst.

Mr.\ Ralf Schellhammer, for providing me with the necessary hardware to conduct the experiments.
His motivation in explaining technical intricacies of radio transmission and his willingness to help me with installing some \ac{LoRaWAN} gateways was invaluable.

Mr.\ Odin Jäger, owner of the \ac{HFU} O building, for providing me with the opportunity to install a gateway on top of that building.

Mr.\ Martin Kramer, for configuring the network I used for the gateways as well as connecting them to the \ac{HFU} network infrastructure.

The club of amateur radio operators based in Furtwangen, \emph{DL0FIS}, for providing me with the opportunity to install a gateway on top of their radio mast in the vicinity of Furtwangen.