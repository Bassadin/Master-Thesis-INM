\chapter*{Abstract\markboth{Abstract}}
\addcontentsline{toc}{chapter}{Abstract}

The localization of \ac{IoT} end devices is a topic of great interest in many use cases.
\ac{GPS} consumes a lot of energy and only works reliably outdoors, making it unsuitable for most \ac{IoT} use cases.
\ac{LoRaWAN} is an \ac{IoT} communication standard with long range and low power consumption.
In this thesis, different types of localization with \ac{LoRaWAN} signal strengths have been tested and evaluated.
Methods using multilateration by \acf{ToA} and \acf{RSSI} as well as fingerprinting with \ac{RSSI} and other values were evaluated.
To efficiently represent some of these localization options, a full-stack web application was developed.
The existing \ac{LoRaWAN} infrastructure was used and several new gateways were installed in the Furtwangen area.

Eine Lokalisierung von \ac{IoT}-Endgeräten ist in vielen Anwendungsfällen von großem Interesse.
\ac{GPS} ist energiehungrig und funktioniert nur unter freiem Himmel zuverlässig, was es unpassend für die meisten \ac{IoT} Usecases macht.
\ac{LoRaWAN} ist ein \ac{IoT}-Kommunikationsstandard mit großer Reichweite und geringem Energieverbrauch.
In dieser Arbeit wurden verschiedene Arten der Lokalisierung mit \ac{LoRaWAN}-Signalstärken erprobt und evaluiert.
Evaluiert wurden die Methoden Multilateration via \acf{ToA} und \acf{RSSI} sowie Fingerprinting mit \ac{RSSI} und anderen Messwerten.
Um einige dieser Lokalisierungsoptionen darzustellen, wurde eine Full-Stack-Webapplikation entwickelt.
Es wurde auf bestehende \ac{LoRaWAN}-Infrastruktur zurückgegriffen und zusätzlich neue Gateways an mehreren Standorten im erweiterten Stadtgebiet von Furtwangen installiert.