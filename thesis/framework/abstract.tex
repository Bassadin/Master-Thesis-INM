\chapter*{Abstract\markboth{Abstract}}
\addcontentsline{toc}{chapter}{Abstract}

Localization of \ac{IoT} end devices is a topic of great interest in many use cases.
The emerging wireless communication standard \ac{LoRa} and its network interconnection technology \ac{LoRaWAN} offer a selection of possibilities for this.
In the context of this thesis, some of those localization possibilities for end devices were tested.
The methods multilateration by Time of Arrival (\acf{ToA}) and \acs{RSSI} as well as fingerprinting with \acs{RSSI} were evaluated.
To efficiently depict some of these localization options, a full-stack web application was developed.
Existing \ac{LoRaWAN} infrastructure was used as well gateways installed in the Furtwangen area.

Eine Lokalisierung von \ac{IoT}-Endgeräten ist in vielen Anwendungsfällen von großem Interesse.
Der aufkommende drahtlose Kommunikationsstandard \ac{LoRa} bzw. seine Netzwerkverbundtechnik \ac{LoRaWAN} bieten hierfür eine Auswahl an Möglichkeiten.
Im Rahmen dieser Thesis wurden jene Möglichkeiten der Lokalisierung von Endgeräten getestet.
Evaluiert wurden die Methoden Multilateration durch Time of Arrival (\acf{ToA}) und \acs{RSSI} sowie Fingerprinting mit \acs{RSSI}.
Um einige dieser Lokalisierungsoptionen effizient darzustellen, wurde eine Full-Stack-Webapplikation entwickelt.
Es wurde auf bestehende \ac{LoRaWAN}-Infrastruktur zurückgegriffen und zusätzlich neue Gateways in Furtwangen installiert.



